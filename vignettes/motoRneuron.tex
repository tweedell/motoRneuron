\documentclass[]{article}
\usepackage{lmodern}
\usepackage{amssymb,amsmath}
\usepackage{ifxetex,ifluatex}
\usepackage{fixltx2e} % provides \textsubscript
\ifnum 0\ifxetex 1\fi\ifluatex 1\fi=0 % if pdftex
  \usepackage[T1]{fontenc}
  \usepackage[utf8]{inputenc}
\else % if luatex or xelatex
  \ifxetex
    \usepackage{mathspec}
  \else
    \usepackage{fontspec}
  \fi
  \defaultfontfeatures{Ligatures=TeX,Scale=MatchLowercase}
\fi
% use upquote if available, for straight quotes in verbatim environments
\IfFileExists{upquote.sty}{\usepackage{upquote}}{}
% use microtype if available
\IfFileExists{microtype.sty}{%
\usepackage{microtype}
\UseMicrotypeSet[protrusion]{basicmath} % disable protrusion for tt fonts
}{}
\usepackage[margin=1in]{geometry}
\usepackage{hyperref}
\hypersetup{unicode=true,
            pdftitle={motoRneuron},
            pdfauthor={Andrew Tweedell},
            pdfborder={0 0 0},
            breaklinks=true}
\urlstyle{same}  % don't use monospace font for urls
\usepackage{color}
\usepackage{fancyvrb}
\newcommand{\VerbBar}{|}
\newcommand{\VERB}{\Verb[commandchars=\\\{\}]}
\DefineVerbatimEnvironment{Highlighting}{Verbatim}{commandchars=\\\{\}}
% Add ',fontsize=\small' for more characters per line
\usepackage{framed}
\definecolor{shadecolor}{RGB}{248,248,248}
\newenvironment{Shaded}{\begin{snugshade}}{\end{snugshade}}
\newcommand{\KeywordTok}[1]{\textcolor[rgb]{0.13,0.29,0.53}{\textbf{#1}}}
\newcommand{\DataTypeTok}[1]{\textcolor[rgb]{0.13,0.29,0.53}{#1}}
\newcommand{\DecValTok}[1]{\textcolor[rgb]{0.00,0.00,0.81}{#1}}
\newcommand{\BaseNTok}[1]{\textcolor[rgb]{0.00,0.00,0.81}{#1}}
\newcommand{\FloatTok}[1]{\textcolor[rgb]{0.00,0.00,0.81}{#1}}
\newcommand{\ConstantTok}[1]{\textcolor[rgb]{0.00,0.00,0.00}{#1}}
\newcommand{\CharTok}[1]{\textcolor[rgb]{0.31,0.60,0.02}{#1}}
\newcommand{\SpecialCharTok}[1]{\textcolor[rgb]{0.00,0.00,0.00}{#1}}
\newcommand{\StringTok}[1]{\textcolor[rgb]{0.31,0.60,0.02}{#1}}
\newcommand{\VerbatimStringTok}[1]{\textcolor[rgb]{0.31,0.60,0.02}{#1}}
\newcommand{\SpecialStringTok}[1]{\textcolor[rgb]{0.31,0.60,0.02}{#1}}
\newcommand{\ImportTok}[1]{#1}
\newcommand{\CommentTok}[1]{\textcolor[rgb]{0.56,0.35,0.01}{\textit{#1}}}
\newcommand{\DocumentationTok}[1]{\textcolor[rgb]{0.56,0.35,0.01}{\textbf{\textit{#1}}}}
\newcommand{\AnnotationTok}[1]{\textcolor[rgb]{0.56,0.35,0.01}{\textbf{\textit{#1}}}}
\newcommand{\CommentVarTok}[1]{\textcolor[rgb]{0.56,0.35,0.01}{\textbf{\textit{#1}}}}
\newcommand{\OtherTok}[1]{\textcolor[rgb]{0.56,0.35,0.01}{#1}}
\newcommand{\FunctionTok}[1]{\textcolor[rgb]{0.00,0.00,0.00}{#1}}
\newcommand{\VariableTok}[1]{\textcolor[rgb]{0.00,0.00,0.00}{#1}}
\newcommand{\ControlFlowTok}[1]{\textcolor[rgb]{0.13,0.29,0.53}{\textbf{#1}}}
\newcommand{\OperatorTok}[1]{\textcolor[rgb]{0.81,0.36,0.00}{\textbf{#1}}}
\newcommand{\BuiltInTok}[1]{#1}
\newcommand{\ExtensionTok}[1]{#1}
\newcommand{\PreprocessorTok}[1]{\textcolor[rgb]{0.56,0.35,0.01}{\textit{#1}}}
\newcommand{\AttributeTok}[1]{\textcolor[rgb]{0.77,0.63,0.00}{#1}}
\newcommand{\RegionMarkerTok}[1]{#1}
\newcommand{\InformationTok}[1]{\textcolor[rgb]{0.56,0.35,0.01}{\textbf{\textit{#1}}}}
\newcommand{\WarningTok}[1]{\textcolor[rgb]{0.56,0.35,0.01}{\textbf{\textit{#1}}}}
\newcommand{\AlertTok}[1]{\textcolor[rgb]{0.94,0.16,0.16}{#1}}
\newcommand{\ErrorTok}[1]{\textcolor[rgb]{0.64,0.00,0.00}{\textbf{#1}}}
\newcommand{\NormalTok}[1]{#1}
\usepackage{graphicx,grffile}
\makeatletter
\def\maxwidth{\ifdim\Gin@nat@width>\linewidth\linewidth\else\Gin@nat@width\fi}
\def\maxheight{\ifdim\Gin@nat@height>\textheight\textheight\else\Gin@nat@height\fi}
\makeatother
% Scale images if necessary, so that they will not overflow the page
% margins by default, and it is still possible to overwrite the defaults
% using explicit options in \includegraphics[width, height, ...]{}
\setkeys{Gin}{width=\maxwidth,height=\maxheight,keepaspectratio}
\IfFileExists{parskip.sty}{%
\usepackage{parskip}
}{% else
\setlength{\parindent}{0pt}
\setlength{\parskip}{6pt plus 2pt minus 1pt}
}
\setlength{\emergencystretch}{3em}  % prevent overfull lines
\providecommand{\tightlist}{%
  \setlength{\itemsep}{0pt}\setlength{\parskip}{0pt}}
\setcounter{secnumdepth}{0}
% Redefines (sub)paragraphs to behave more like sections
\ifx\paragraph\undefined\else
\let\oldparagraph\paragraph
\renewcommand{\paragraph}[1]{\oldparagraph{#1}\mbox{}}
\fi
\ifx\subparagraph\undefined\else
\let\oldsubparagraph\subparagraph
\renewcommand{\subparagraph}[1]{\oldsubparagraph{#1}\mbox{}}
\fi

%%% Use protect on footnotes to avoid problems with footnotes in titles
\let\rmarkdownfootnote\footnote%
\def\footnote{\protect\rmarkdownfootnote}

%%% Change title format to be more compact
\usepackage{titling}

% Create subtitle command for use in maketitle
\newcommand{\subtitle}[1]{
  \posttitle{
    \begin{center}\large#1\end{center}
    }
}

\setlength{\droptitle}{-2em}

  \title{motoRneuron}
    \pretitle{\vspace{\droptitle}\centering\huge}
  \posttitle{\par}
    \author{Andrew Tweedell}
    \preauthor{\centering\large\emph}
  \postauthor{\par}
      \predate{\centering\large\emph}
  \postdate{\par}
    \date{September 19, 2018}


\begin{document}
\maketitle

\section{Using motoRneuron for motor unit
synchronization}\label{using-motorneuron-for-motor-unit-synchronization}

The purpose of this vignette is get you started using the
\texttt{motoRneuron} package for analyzing time-domain motor unit
synchronization, even if you've never used R before! So we will show you
how to install the toolbox through CRAN, access the functions, help
files, and sample motor unit data. Next, we will provide code to get
typical time series data into a workable format for our functions. Then
we will go step by step through some analyses with this sample data to
demonstrate its applicability.

\subsection{Time-domain
synchronization}\label{time-domain-synchronization}

It's assumed that if you're reading this, you have at least a grasp of
what motor unit synchronization means and what you hope to get out of
this package. But breifly, time-domain synchronization, in the
neuromuscular context we are presenting it, is the tendency of separate
motor units (i.e.~motor neurons and their associated muscle fibers) to
discharge near-simultaneously (within 1 - 5 ms of each other) more often
than would be expected by chance. It is often interpreted as an
indicator of functional connectivity between motor neurons through
common excitatory post-synaptic potentials. This is quanitified by a
cross-correlation analyses, which is essentially the core of this
package. A peak in the resulting cross-correlation histogram represents
a higher probability of discharges at that latency. Indices that
quantify the magnitude of the peak are the primary outputs for the
package

\begin{enumerate}
\def\labelenumi{\arabic{enumi})}
\tightlist
\item
  \texttt{recurrence\_intervals()} correlates the discharge times of one
  motor unit against those of another with motor unit.
\item
  \texttt{bin()} creates cross-correlation histogram from the recurrence
  intervals.
\item
  \texttt{mu\_synch()} is an all-in-one function that automatically
  performs the cross-correlation process above and assesses the
  histogram for peaks. 6 commonly used synchronization indices are
  returned.
\item
  3 methods of peak determination are available as individual functions
  as well.

  \begin{enumerate}
  \def\labelenumii{\arabic{enumii})}
  \tightlist
  \item
    \texttt{visual\_mu\_synch()} - Visual method
  \item
    \texttt{cumsum\_mu\_synch()} - Cumulative Sum (Cumsum) method
  \item
    \texttt{zscore\_mu\_synch()} - Zscore method
  \end{enumerate}
\end{enumerate}

\subsection{Package Implementation}\label{package-implementation}

Install the package off the Comprehensive R Archive Network and attach
it to your workspace with the following functions.

\begin{verbatim}
install.packages("motoRneuron")
\end{verbatim}

\begin{Shaded}
\begin{Highlighting}[]
\KeywordTok{library}\NormalTok{(motoRneuron)}
\end{Highlighting}
\end{Shaded}

View the package or any of the function help files by using `?'.

\begin{verbatim}
?motoRneuron
?mu_synch
\end{verbatim}

\subsection{Data: motor\_unit\_data}\label{data-motor_unit_data}

Let's load the sample motor unit data frame from the package into the
working environment and look at it before we play around with it. The
data is 30 seconds of 2 concurrently active motor units discharging at
semi-regular intervals sampled at 1000 Hz.

\begin{Shaded}
\begin{Highlighting}[]
\NormalTok{motor_unit_data <-}\StringTok{ }\NormalTok{motoRneuron}\OperatorTok{::}\NormalTok{motor_unit_data}
\KeywordTok{head}\NormalTok{(motor_unit_data)}
\end{Highlighting}
\end{Shaded}

\begin{verbatim}
##    Time motor_unit_1 motor_unit_2
## 1 0.000            0            0
## 2 0.001            0            0
## 3 0.002            0            0
## 4 0.003            0            0
## 5 0.004            0            0
## 6 0.005            0            0
\end{verbatim}

You'll see our first column is Time in seconds. Makes sense. The other
columns indicate what motor units are active with 1's indicating a
discharge at that time point. But the synchronization functions say they
want the actual discharge times as inputs\ldots{} so how do I get that?
We'll help you out. We included this step because this is a commonly
used format for time series data like motor unit discharges. So to get
the data in a usable form for our functions, we will have to subset out
only the time points the motor unit has a 1.

\begin{Shaded}
\begin{Highlighting}[]
\NormalTok{motor_unit_}\DecValTok{1}\NormalTok{ <-}\StringTok{ }\KeywordTok{data.frame}\NormalTok{(}\DataTypeTok{Time =}\NormalTok{ motor_unit_data[[}\StringTok{"Time"}\NormalTok{]], }\DataTypeTok{MotorUnit1 =}\NormalTok{ motor_unit_data[[}\DecValTok{2}\NormalTok{]]) }\OperatorTok\StringTok{ }
\StringTok{  }\KeywordTok{subset}\NormalTok{(. , MotorUnit1 }\OperatorTok{==}\DecValTok{1}\NormalTok{) }
\NormalTok{motor_unit_}\DecValTok{1}\NormalTok{ <-}\StringTok{ }\KeywordTok{as.vector}\NormalTok{(motor_unit_}\DecValTok{1}\OperatorTok{$}\NormalTok{Time)}
\NormalTok{motor_unit_}\DecValTok{2}\NormalTok{ <-}\StringTok{ }\KeywordTok{data.frame}\NormalTok{(}\DataTypeTok{Time =}\NormalTok{ motor_unit_data[[}\StringTok{"Time"}\NormalTok{]], }\DataTypeTok{MotorUnit2 =}\NormalTok{ motor_unit_data[[}\DecValTok{3}\NormalTok{]]) }\OperatorTok
\StringTok{  }\KeywordTok{subset}\NormalTok{(. , MotorUnit2 }\OperatorTok{==}\DecValTok{1}\NormalTok{)}
\NormalTok{motor_unit_}\DecValTok{2}\NormalTok{ <-}\StringTok{ }\KeywordTok{as.vector}\NormalTok{(motor_unit_}\DecValTok{2}\OperatorTok{$}\NormalTok{Time)}
\end{Highlighting}
\end{Shaded}

You'll see the individual motor unit data is now in vector format
containing only the time points of discharge.

\begin{Shaded}
\begin{Highlighting}[]
\KeywordTok{head}\NormalTok{(motor_unit_}\DecValTok{1}\NormalTok{)}
\end{Highlighting}
\end{Shaded}

\begin{verbatim}
## [1] 0.035 0.115 0.183 0.250 0.306 0.377
\end{verbatim}

\begin{Shaded}
\begin{Highlighting}[]
\KeywordTok{head}\NormalTok{(motor_unit_}\DecValTok{2}\NormalTok{)}
\end{Highlighting}
\end{Shaded}

\begin{verbatim}
## [1] 0.100 0.205 0.298 0.377 0.471 0.577
\end{verbatim}

So we've got our data in a format that are \texttt{motoRneuron}
functions will take.

\subsection{\texorpdfstring{Calculating
\texttt{recurrence\_intervals()}}{Calculating recurrence\_intervals()}}\label{calculating-recurrence_intervals}

\texttt{recurrence\_intervals()} calculates the recurrence intervals
based on how many \emph{recurrence interval orders} is specified (see
image below). Specifically they calculated as the timing difference
between a discharge event in one (reference) motor unit and the nearest
forward and nearest backward discharge events in another (event) motor
unit. The motor unit used as the reference unit is the unit with
\emph{fewer} discharges. Order refers to how many discharges before and
after we are calculating. Default settings are for first order
recurrence intervals only as previous research has shown false peaks are
likely with higher orders.

\includegraphics[width=1\linewidth]{C:/Users/andrew.tweedell.ctr/Desktop/Synch Working Folder/R Package Creation/motoRneuron/images/Recurrence_Intervals_D iagram}
The function takes the vector arguments motor\_unit\_1 and
motr\_unit\_2, along with a numeric indicating the order. The code below
only returns the first forward and first backward discharges.

\begin{Shaded}
\begin{Highlighting}[]
\NormalTok{recur <-}\StringTok{ }\KeywordTok{recurrence_intervals}\NormalTok{(motor_unit_}\DecValTok{1}\NormalTok{, motor_unit_}\DecValTok{2}\NormalTok{, }\DataTypeTok{order =} \DecValTok{1}\NormalTok{)}
\NormalTok{recur}
\end{Highlighting}
\end{Shaded}

\begin{verbatim}
## $Reference_Unit
## [1] "motor_unit_2"
## 
## $Number_of_Reference_Discharges
## [1] 307
## 
## $Reference_ISI
##   [1] 0.105 0.093 0.079 0.094 0.106 0.087 0.075 0.104 0.102 0.101 0.113
##  [12] 0.125 0.082 0.096 0.096 0.096 0.108 0.077 0.107 0.091 0.094 0.089
##  [23] 0.081 0.095 0.035 0.087 0.098 0.112 0.094 0.106 0.087 0.102 0.095
##  [34] 0.101 0.084 0.106 0.078 0.089 0.096 0.099 0.077 0.075 0.091 0.100
##  [45] 0.105 0.099 0.128 0.129 0.076 0.091 0.129 0.087 0.099 0.120 0.126
##  [56] 0.098 0.050 0.154 0.106 0.116 0.108 0.103 0.082 0.149 0.116 0.110
##  [67] 0.116 0.088 0.072 0.068 0.097 0.114 0.076 0.103 0.097 0.091 0.141
##  [78] 0.122 0.081 0.095 0.101 0.101 0.150 0.101 0.074 0.126 0.084 0.169
##  [89] 0.095 0.108 0.100 0.106 0.118 0.097 0.111 0.086 0.118 0.090 0.079
## [100] 0.095 0.126 0.107 0.148 0.094 0.092 0.142 0.146 0.082 0.091 0.089
## [111] 0.088 0.153 0.084 0.076 0.105 0.088 0.144 0.075 0.127 0.095 0.105
## [122] 0.081 0.090 0.069 0.103 0.077 0.088 0.092 0.096 0.106 0.136 0.059
## [133] 0.103 0.086 0.102 0.094 0.093 0.121 0.111 0.063 0.083 0.090 0.123
## [144] 0.099 0.048 0.101 0.061 0.058 0.066 0.089 0.035 0.073 0.070 0.085
## [155] 0.095 0.144 0.104 0.097 0.098 0.118 0.095 0.083 0.079 0.076 0.082
## [166] 0.078 0.137 0.093 0.076 0.105 0.065 0.072 0.089 0.112 0.081 0.090
## [177] 0.125 0.060 0.097 0.127 0.081 0.101 0.096 0.115 0.091 0.093 0.111
## [188] 0.086 0.089 0.099 0.081 0.097 0.087 0.093 0.096 0.084 0.078 0.089
## [199] 0.071 0.088 0.099 0.088 0.117 0.110 0.079 0.162 0.095 0.096 0.078
## [210] 0.077 0.062 0.078 0.083 0.089 0.098 0.176 0.123 0.149 0.076 0.145
## [221] 0.102 0.086 0.111 0.138 0.113 0.091 0.124 0.089 0.088 0.153 0.085
## [232] 0.088 0.091 0.097 0.142 0.099 0.089 0.080 0.096 0.072 0.106 0.071
## [243] 0.111 0.106 0.093 0.101 0.086 0.098 0.077 0.095 0.099 0.092 0.170
## [254] 0.091 0.103 0.107 0.107 0.092 0.083 0.106 0.120 0.080 0.065 0.186
## [265] 0.125 0.076 0.100 0.108 0.077 0.057 0.063 0.104 0.078 0.089 0.175
## [276] 0.074 0.072 0.116 0.097 0.074 0.090 0.121 0.095 0.103 0.092 0.083
## [287] 0.095 0.089 0.106 0.077 0.082 0.093 0.092 0.106 0.102 0.086 0.094
## [298] 0.093 0.095 0.107 0.112 0.092 0.116 0.066 0.090 0.093
## 
## $Mean_Reference_ISI
## [1] 0.098
## 
## $Event_Unit
## [1] "motor_unit_1"
## 
## $Number_of_Event_Discharges
## [1] 443
## 
## $Event_ISI
##   [1] 0.080 0.068 0.067 0.056 0.071 0.078 0.057 0.065 0.079 0.083 0.082
##  [12] 0.045 0.084 0.064 0.071 0.068 0.060 0.066 0.076 0.076 0.051 0.069
##  [23] 0.080 0.061 0.057 0.083 0.065 0.067 0.066 0.076 0.043 0.080 0.073
##  [34] 0.082 0.036 0.085 0.087 0.052 0.078 0.079 0.069 0.056 0.085 0.087
##  [45] 0.085 0.069 0.067 0.076 0.074 0.091 0.029 0.077 0.078 0.084 0.048
##  [56] 0.067 0.077 0.084 0.054 0.058 0.082 0.083 0.032 0.087 0.064 0.089
##  [67] 0.083 0.052 0.061 0.083 0.081 0.065 0.056 0.099 0.050 0.071 0.092
##  [78] 0.064 0.076 0.099 0.062 0.085 0.088 0.066 0.055 0.084 0.066 0.069
##  [89] 0.077 0.047 0.088 0.082 0.077 0.071 0.068 0.082 0.081 0.052 0.086
## [100] 0.052 0.089 0.053 0.068 0.088 0.091 0.052 0.089 0.086 0.056 0.062
## [111] 0.093 0.073 0.053 0.072 0.066 0.070 0.048 0.064 0.075 0.021 0.091
## [122] 0.082 0.066 0.067 0.062 0.085 0.050 0.069 0.067 0.074 0.073 0.078
## [133] 0.049 0.070 0.075 0.071 0.055 0.087 0.073 0.049 0.085 0.067 0.074
## [144] 0.073 0.059 0.070 0.055 0.071 0.087 0.054 0.068 0.071 0.058 0.083
## [155] 0.095 0.046 0.079 0.074 0.077 0.071 0.056 0.064 0.082 0.066 0.076
## [166] 0.062 0.060 0.076 0.066 0.062 0.057 0.085 0.066 0.056 0.067 0.064
## [177] 0.044 0.072 0.048 0.067 0.062 0.065 0.082 0.105 0.027 0.080 0.086
## [188] 0.040 0.065 0.086 0.025 0.084 0.068 0.065 0.058 0.068 0.041 0.082
## [199] 0.045 0.079 0.066 0.078 0.062 0.052 0.083 0.049 0.068 0.056 0.069
## [210] 0.050 0.069 0.071 0.073 0.062 0.068 0.048 0.074 0.059 0.084 0.028
## [221] 0.080 0.068 0.079 0.035 0.067 0.089 0.043 0.068 0.076 0.048 0.067
## [232] 0.073 0.080 0.073 0.065 0.046 0.069 0.064 0.077 0.060 0.054 0.070
## [243] 0.061 0.068 0.047 0.061 0.080 0.050 0.062 0.072 0.050 0.062 0.079
## [254] 0.072 0.050 0.066 0.065 0.069 0.068 0.046 0.069 0.081 0.031 0.081
## [265] 0.087 0.050 0.057 0.060 0.068 0.072 0.076 0.058 0.059 0.066 0.061
## [276] 0.061 0.072 0.083 0.034 0.078 0.068 0.077 0.047 0.067 0.062 0.077
## [287] 0.054 0.054 0.082 0.084 0.054 0.067 0.073 0.047 0.073 0.070 0.088
## [298] 0.040 0.080 0.068 0.073 0.053 0.079 0.071 0.045 0.068 0.079 0.057
## [309] 0.080 0.080 0.074 0.060 0.083 0.077 0.081 0.058 0.065 0.070 0.083
## [320] 0.030 0.084 0.073 0.071 0.078 0.046 0.072 0.069 0.065 0.078 0.079
## [331] 0.080 0.078 0.079 0.075 0.057 0.064 0.074 0.063 0.059 0.059 0.072
## [342] 0.060 0.060 0.070 0.085 0.034 0.041 0.052 0.075 0.066 0.072 0.061
## [353] 0.072 0.083 0.033 0.078 0.079 0.073 0.088 0.051 0.060 0.091 0.047
## [364] 0.066 0.069 0.076 0.053 0.080 0.070 0.064 0.050 0.088 0.074 0.060
## [375] 0.057 0.076 0.068 0.061 0.072 0.077 0.064 0.065 0.075 0.078 0.052
## [386] 0.070 0.076 0.065 0.071 0.066 0.064 0.063 0.079 0.055 0.060 0.065
## [397] 0.072 0.070 0.078 0.067 0.047 0.088 0.050 0.063 0.067 0.073 0.071
## [408] 0.073 0.052 0.063 0.075 0.073 0.045 0.074 0.074 0.050 0.061 0.080
## [419] 0.070 0.063 0.060 0.078 0.102 0.073 0.066 0.062 0.061 0.068 0.050
## [430] 0.068 0.047 0.079 0.066 0.069 0.078 0.069 0.062 0.069 0.070 0.063
## [441] 0.061 0.064
## 
## $Mean_Event_ISI
## [1] 0.068
## 
## $Duration
## [1] 29.95
## 
## $`1`
##   [1] -0.065  0.015 -0.022  0.045 -0.048  0.008 -0.071  0.000 -0.016  0.041
##  [11] -0.065  0.000 -0.008  0.075 -0.083  0.000 -0.022  0.023 -0.079  0.005
##  [21] -0.032  0.039 -0.006  0.054 -0.005  0.071 -0.011  0.065 -0.031  0.020
##  [31] -0.007  0.073 -0.023  0.038 -0.013  0.070 -0.007  0.058 -0.049  0.018
##  [41] -0.007  0.069 -0.025  0.018 -0.071  0.009 -0.072  0.001 -0.012  0.024
##  [51] -0.011  0.074 -0.013  0.074 -0.024  0.028 -0.006  0.073 -0.021  0.048
##  [61] -0.002  0.083 -0.004  0.083 -0.019  0.066 -0.029  0.040 -0.061  0.006
##  [71] -0.002  0.072 -0.034  0.057 -0.021  0.008 -0.004  0.074 -0.022  0.062
##  [81] -0.037  0.011 -0.066  0.001 -0.074  0.003 -0.004  0.050 -0.050  0.008
##  [91] -0.015  0.068 -0.031  0.001 -0.040  0.024 -0.016  0.067 -0.009  0.043
## [101] -0.048  0.013 -0.033  0.048 -0.039  0.026 -0.017  0.082 -0.038  0.012
## [111] -0.043  0.049 -0.049  0.015 -0.035  0.041 -0.014  0.048 -0.058  0.027
## [121] -0.001  0.065 -0.043  0.012 -0.007  0.059 -0.023  0.046 -0.026  0.021
## [131] -0.007  0.075 -0.035  0.042 -0.003  0.065 -0.023  0.059 -0.013  0.068
## [141] -0.081  0.000 -0.045  0.041 -0.021  0.068 -0.008  0.045 -0.058  0.010
## [151] -0.087  0.001 -0.090  0.001 -0.088  0.001 -0.035  0.021 -0.060  0.002
## [161] -0.093  0.000 -0.028  0.025 -0.004  0.062 -0.018  0.030 -0.007  0.068
## [171] -0.006  0.015 -0.020  0.062 -0.022  0.044 -0.058  0.004 -0.006  0.044
## [181] -0.064  0.005 -0.028  0.046 -0.060  0.013 -0.027  0.022 -0.005  0.070
## [191] -0.041  0.030 -0.001  0.086 -0.032  0.041 -0.049  0.000 -0.079  0.006
## [201] -0.022  0.052 -0.001  0.058 -0.049  0.021 -0.001  0.086 -0.008  0.046
## [211] -0.046  0.022 -0.049  0.009 -0.054  0.041 -0.041  0.005 -0.007  0.067
## [221] -0.022  0.055 -0.033  0.038 -0.059  0.005 -0.079  0.003 -0.007  0.069
## [231] -0.036  0.026 -0.002  0.074 -0.004  0.058 -0.017  0.040 -0.002  0.064
## [241] -0.031  0.025 -0.013  0.051 -0.030  0.014 -0.004  0.044 -0.025  0.042
## [251] -0.061  0.001 -0.011  0.071 -0.017  0.088 -0.004  0.023 -0.073  0.007
## [261] -0.013  0.027 -0.044  0.042 -0.017  0.008 -0.011  0.057 -0.029  0.036
## [271] -0.008  0.060 -0.034  0.007 -0.004  0.041 -0.001  0.065 -0.046  0.032
## [281] -0.031  0.031 -0.052  0.000 -0.007  0.042 -0.013  0.043 -0.056  0.013
## [291] -0.035  0.015 -0.017  0.054 -0.007  0.066 -0.065  0.008 -0.058  0.004
## [301] -0.017  0.031 -0.004  0.070 -0.003  0.056 -0.014  0.070 -0.015  0.013
## [311] -0.002  0.066 -0.078  0.001 -0.001  0.088 -0.009  0.034 -0.064  0.004
## [321] -0.038  0.010 -0.018  0.055 -0.028  0.052 -0.027  0.046 -0.030  0.035
## [331] -0.001  0.068 -0.010  0.054 -0.006  0.054 -0.039  0.015 -0.061  0.009
## [341] -0.035  0.033 -0.032  0.015 -0.057  0.004 -0.005  0.045 -0.005  0.067
## [351] -0.014  0.036 -0.054  0.008 -0.038  0.034 -0.026  0.024 -0.007  0.058
## [361] -0.069  0.000 -0.013  0.033 -0.068  0.001 -0.014  0.017 -0.017  0.070
## [371] -0.021  0.029 -0.007  0.053 -0.058  0.010 -0.004  0.072 -0.017  0.041
## [381] -0.058  0.001 -0.014  0.047 -0.050  0.011 -0.004  0.079 -0.014  0.020
## [391] -0.076  0.002 -0.014  0.063 -0.015  0.032 -0.057  0.010 -0.061  0.001
## [401] -0.010  0.044 -0.001  0.081 -0.007  0.077 -0.040  0.014 -0.029  0.044
## [411] -0.035  0.012 -0.007  0.081 -0.014  0.026 -0.070  0.010 -0.068  0.000
## [421] -0.004  0.049 -0.013  0.066 -0.012  0.059 -0.024  0.021 -0.068  0.000
## [431] -0.019  0.038 -0.058  0.022 -0.027  0.033 -0.033  0.044 -0.032  0.049
## [441] -0.038  0.027 -0.005  0.078 -0.008  0.022 -0.005  0.068 -0.070  0.001
## [451] -0.034  0.012 -0.007  0.062 -0.062  0.003 -0.008  0.071 -0.017  0.063
## [461] -0.012  0.067 -0.018  0.057 -0.031  0.026 -0.001  0.073 -0.024  0.039
## [471] -0.044  0.015 -0.012  0.048 -0.041  0.019 -0.061  0.009 -0.002  0.032
## [481] -0.040  0.001 -0.053  0.022 -0.049  0.017 -0.022  0.039 -0.067  0.005
## [491] -0.005  0.028 -0.073  0.005 -0.002  0.071 -0.027  0.061 -0.016  0.035
## [501] -0.060  0.000 -0.008  0.039 -0.053  0.013 -0.012  0.041 -0.050  0.030
## [511] -0.003  0.061 -0.046  0.004 -0.015  0.059 -0.033  0.027 -0.056  0.001
## [521] -0.029  0.039 -0.020  0.052 -0.028  0.049 -0.016  0.048 -0.073  0.002
## [531] -0.045  0.007 -0.069  0.001 -0.023  0.042 -0.066  0.005 -0.006  0.058
## [541] -0.063  0.001 -0.062  0.001 -0.024  0.031 -0.047  0.013 -0.011  0.061
## [551] -0.044  0.034 -0.040  0.027 -0.045  0.002 -0.026  0.024 -0.010  0.057
## [561] -0.017  0.056 -0.034  0.037 -0.011  0.041 -0.054  0.009 -0.019  0.054
## [571] -0.038  0.007 -0.002  0.072 -0.023  0.027 -0.001  0.079 -0.027  0.043
## [581] -0.034  0.029 -0.053  0.007 -0.008  0.094 -0.100  0.002 -0.031  0.035
## [591] -0.005  0.056 -0.030  0.038 -0.006  0.062 -0.031  0.016 -0.079  0.000
## [601] -0.041  0.028 -0.006  0.063 -0.029  0.033 -0.014  0.056 -0.010  0.053
## [611] -0.037  0.024 -0.005
\end{verbatim}

A list is returned that contains motor unit characteristics and at the
end is a vector containing all the recurrence intervals.

\subsection{\texorpdfstring{Discretize recurrence intervals with
\texttt{bin()}}{Discretize recurrence intervals with bin()}}\label{discretize-recurrence-intervals-with-bin}

Only 2 arguments are needed for \texttt{bin()}; a vector containing the
recurrence intervals and a numeric indicating the bin size or bin width.
Using \texttt{unlist()} will take the intervals from before out a list
format and into a vector format.

\begin{Shaded}
\begin{Highlighting}[]
\NormalTok{recur <-}\StringTok{ }\KeywordTok{unlist}\NormalTok{(recur}\OperatorTok{$}\StringTok{`}\DataTypeTok{1}\StringTok{`}\NormalTok{)}
\NormalTok{binned_data <-}\StringTok{ }\KeywordTok{bin}\NormalTok{(recur, }\DataTypeTok{binwidth =} \FloatTok{0.001}\NormalTok{)}
\NormalTok{binned_data}
\end{Highlighting}
\end{Shaded}

\begin{verbatim}
##        Bin Freq
## 1   -0.101    1
## 2   -0.100    0
## 3   -0.099    0
## 4   -0.098    0
## 5   -0.097    0
## 6   -0.096    0
## 7   -0.095    0
## 8   -0.094    0
## 9   -0.093    1
## 10  -0.092    0
## 11  -0.091    0
## 12  -0.090    1
## 13  -0.089    1
## 14  -0.088    0
## 15  -0.087    1
## 16  -0.086    0
## 17  -0.085    0
## 18  -0.084    0
## 19  -0.083    1
## 20  -0.082    0
## 21  -0.081    1
## 22  -0.080    2
## 23  -0.079    3
## 24  -0.078    0
## 25  -0.077    1
## 26  -0.076    0
## 27  -0.075    1
## 28  -0.074    2
## 29  -0.073    2
## 30  -0.072    1
## 31  -0.071    3
## 32  -0.070    1
## 33  -0.069    2
## 34  -0.068    3
## 35  -0.067    1
## 36  -0.066    2
## 37  -0.065    4
## 38  -0.064    0
## 39  -0.063    1
## 40  -0.062    3
## 41  -0.061    6
## 42  -0.060    1
## 43  -0.059    2
## 44  -0.058    8
## 45  -0.057    0
## 46  -0.056    2
## 47  -0.055    3
## 48  -0.054    0
## 49  -0.053    3
## 50  -0.052    1
## 51  -0.051    2
## 52  -0.050    3
## 53  -0.049    5
## 54  -0.048    2
## 55  -0.047    0
## 56  -0.046    5
## 57  -0.045    3
## 58  -0.044    3
## 59  -0.043    0
## 60  -0.042    3
## 61  -0.041    2
## 62  -0.040    3
## 63  -0.039    5
## 64  -0.038    3
## 65  -0.037    1
## 66  -0.036    7
## 67  -0.035    3
## 68  -0.034    3
## 69  -0.033    7
## 70  -0.032    2
## 71  -0.031    6
## 72  -0.030    3
## 73  -0.029    7
## 74  -0.028    4
## 75  -0.027    3
## 76  -0.026    2
## 77  -0.025    5
## 78  -0.024    3
## 79  -0.023    7
## 80  -0.022    4
## 81  -0.021    4
## 82  -0.020    3
## 83  -0.019    2
## 84  -0.018    5
## 85  -0.017   10
## 86  -0.016    3
## 87  -0.015    4
## 88  -0.014    7
## 89  -0.013   12
## 90  -0.012    5
## 91  -0.011    3
## 92  -0.010    4
## 93  -0.009    5
## 94  -0.008   12
## 95  -0.007   14
## 96  -0.006    5
## 97  -0.005   10
## 98  -0.004   10
## 99  -0.003    2
## 100 -0.002   13
## 101 -0.001    3
## 102  0.000   21
## 103  0.001   13
## 104  0.002    1
## 105  0.003    5
## 106  0.004    4
## 107  0.005    6
## 108  0.006    4
## 109  0.007    5
## 110  0.008    7
## 111  0.009    4
## 112  0.010    4
## 113  0.011    0
## 114  0.012    9
## 115  0.013    2
## 116  0.014    4
## 117  0.015    4
## 118  0.016    2
## 119  0.017    1
## 120  0.018    2
## 121  0.019    2
## 122  0.020    3
## 123  0.021    4
## 124  0.022    4
## 125  0.023    2
## 126  0.024    4
## 127  0.025    5
## 128  0.026    2
## 129  0.027    6
## 130  0.028    4
## 131  0.029    1
## 132  0.030    2
## 133  0.031    3
## 134  0.032    5
## 135  0.033    4
## 136  0.034    1
## 137  0.035    5
## 138  0.036    1
## 139  0.037    0
## 140  0.038    8
## 141  0.039    3
## 142  0.040    1
## 143  0.041   11
## 144  0.042    3
## 145  0.043    4
## 146  0.044    5
## 147  0.045    5
## 148  0.046    1
## 149  0.047    3
## 150  0.048    7
## 151  0.049    0
## 152  0.050    1
## 153  0.051    3
## 154  0.052    2
## 155  0.053    2
## 156  0.054    5
## 157  0.055    2
## 158  0.056    5
## 159  0.057    5
## 160  0.058    2
## 161  0.059    4
## 162  0.060    4
## 163  0.061    2
## 164  0.062    4
## 165  0.063    2
## 166  0.064    5
## 167  0.065    3
## 168  0.066    1
## 169  0.067    8
## 170  0.068    3
## 171  0.069    2
## 172  0.070    8
## 173  0.071    2
## 174  0.072    3
## 175  0.073    3
## 176  0.074    3
## 177  0.075    1
## 178  0.076    1
## 179  0.077    0
## 180  0.078    2
## 181  0.079    1
## 182  0.080    2
## 183  0.081    1
## 184  0.082    1
## 185  0.083    1
## 186  0.084    0
## 187  0.085    0
## 188  0.086    2
## 189  0.087    0
## 190  0.088    2
## 191  0.089    0
## 192  0.090    0
## 193  0.091    0
## 194  0.092    0
## 195  0.093    0
## 196  0.094    1
\end{verbatim}

A simple data frame is returned. \texttt{Bin} tells us how far before or
after the discharge of our reference motor unit that the other motor
unit discharged. Bin ``0.000'' represents the two motor units firing at
the exact same time. \texttt{Freq} is the frequency of that interval
happening in the trial.

\subsection{\texorpdfstring{Visualization of histogram with
\texttt{plot\_bins()}}{Visualization of histogram with plot\_bins()}}\label{visualization-of-histogram-with-plot_bins}

\texttt{plot\_bins()} leverages ggplot2 to graph the binned recurrence
intervals as a histogram. Mean bin size or average frequency of the bins
is indicated as a red line on the graph.

\begin{Shaded}
\begin{Highlighting}[]
\KeywordTok{plot_bins}\NormalTok{(binned_data)}
\end{Highlighting}
\end{Shaded}

\includegraphics{motoRneuron_files/figure-latex/unnamed-chunk-8-1.pdf}
From the graph, we can see that there are higher frequencies the nearer
to ``0.000'' we get. This may indicate that there is a more than likely
chance that these two motor units are functionally connected somehow. If
there is a peak in this histogram, then our synchronization functions
will determine where they are and calculate the various indices from it.

\subsection{Synchronization functions}\label{synchronization-functions}

\texttt{mu\_synch()} is the all-in-one function that can call 3
different methods for determining if and where there is a peak. Both
\texttt{recurrence\_intervals()} and \texttt{bin()} are called within
\texttt{mu\_synch()} to perform the initial analysis. ``visual'',
``Cumsum'', and ``Zscore'' methods are available for peak determination,
but can also be called individually.

\subsubsection{Visual Determination}\label{visual-determination}

\begin{verbatim}
visual_mu_synch(motor_unit_1, motor_unit_2, order = 1, binwidth = 0.001, plot = F)
\end{verbatim}

\begin{verbatim}
cumsum_mu_synch(motor_unit_1, motor_unit_2, order = 1, binwidth = 0.001, plot = F)
zscore_mu_synch(motor_unit_1, motor_unit_2, order = 1, binwidth = 0.001, plot = F)
\end{verbatim}

Motor unit vectors, order numeric, and binwidth numeric are the same
arguments used for \texttt{recurrence\_intervals()} and \texttt{bin()}.
Plot is a TRUE/FALSE logical on whether to output the histogram from
\texttt{plot\_bins()}.

\begin{Shaded}
\begin{Highlighting}[]
\NormalTok{synch_data <-}\StringTok{ }\KeywordTok{mu_synch}\NormalTok{(motor_unit_}\DecValTok{1}\NormalTok{, motor_unit_}\DecValTok{2}\NormalTok{, }\DataTypeTok{method =} \KeywordTok{c}\NormalTok{(}\StringTok{"Cumsum"}\NormalTok{, }\StringTok{"Zscore"}\NormalTok{), }\DataTypeTok{order =} \DecValTok{1}\NormalTok{, }\DataTypeTok{binwidth =} \FloatTok{0.001}\NormalTok{, }\DataTypeTok{plot =}\NormalTok{ F)}
\end{Highlighting}
\end{Shaded}

The first half of the results for these functions are just the same
motor unit characteristics returned by \texttt{recurrence\_intervals()}.
Here we will only look at the synchronization indices portion of the
data

\begin{Shaded}
\begin{Highlighting}[]
\NormalTok{synch_data}\OperatorTok{$}\StringTok{`}\DataTypeTok{Visual Indices}\StringTok{`}
\end{Highlighting}
\end{Shaded}

\begin{verbatim}
## NULL
\end{verbatim}

\begin{Shaded}
\begin{Highlighting}[]
\NormalTok{synch_data}\OperatorTok{$}\StringTok{`}\DataTypeTok{Cumsum Indices}\StringTok{`}
\end{Highlighting}
\end{Shaded}

\begin{verbatim}
## $CIS
## [1] 2.163168
## 
## $kprime
## [1] 3.79096
## 
## $kminus1
## [1] 2.79096
## 
## $E
## [1] 0.2110322
## 
## $S
## [1] 0.08638251
## 
## $SI
## [1] 0.2117218
## 
## $Peak.duration
## [1] 0.01
## 
## $Peak.center
## [1] 0
\end{verbatim}

\begin{Shaded}
\begin{Highlighting}[]
\NormalTok{synch_data}\OperatorTok{$}\StringTok{`}\DataTypeTok{Zscore Indices}\StringTok{`}
\end{Highlighting}
\end{Shaded}

\begin{verbatim}
## $CIS
## [1] 1.714933
## 
## $kprime
## [1] 4.284502
## 
## $kminus1
## [1] 3.284502
## 
## $E
## [1] 0.1673037
## 
## $S
## [1] 0.06848299
## 
## $SI
## [1] 0.1678505
## 
## $Peak.duration
## [1] NA
## 
## $Peak.center
## [1] NA
\end{verbatim}

Using the ``Visual'' method requires a bit more input by the user.
\texttt{motoRneuron} leverages dygraphs to interactively plot the
normalized cumulative sum of the histogram. This allows the user to
visually pick out the boundaries of the peak, if there is one.

\begin{verbatim}
visual_synch_results <- mu_synch(motor_unit_1, motor_unit_2, method = "Visual", order = 1, binwidth = 0.001, plot = F)
\end{verbatim}


\end{document}
